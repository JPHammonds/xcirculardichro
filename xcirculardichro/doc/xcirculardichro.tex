
% !BIB TS-program = biber
% !BIB program = biber

\documentclass[12pt,letterpaper, openany]{book}
\usepackage[utf8]{inputenc}
\usepackage{titlesec}
\usepackage[cc]{titlepic}
\usepackage[english]{babel}
\usepackage[autostyle]{csquotes}
\usepackage[style=numeric, backend=biber]{biblatex}
\usepackage[]{hyperref}
\usepackage[usenames, dvipsnames]{xcolor}
\usepackage{framed}
\usepackage{realboxes}
\usepackage[toc,page]{appendix}
\definecolor{shadecolor}{rgb}{1,0.8,0.3}
\usepackage{tikz}
\usepackage{graphicx}
\usepackage{subfig}
\usepackage{amsmath}
\usepackage{listings}

\bibliography{xcirculardichro}
\begin{document}
\begin{titlepage}


\titlepic{\includegraphics[width=2in]{images/xmcdchiro.png}}
\title{xcirculardichro Getting Started}
\author{John Hammonds \\
Scientific Software Engineering and Data Management \\
X-Ray Science Division \\
Argonne National Laboratory}

\titlepic{\includegraphics[width=2in]{images/ANL_RGB-01.png}}

\end{titlepage}

\maketitle
\tableofcontents
\listoffigures
\chapter{Introduction}\label{chap:introduction}

\section{Purpose}\label{sec:purpose}
This program \textbf{\textit{xcirculardichro}} is intended to make it easier to
get results from X-ray Circular Magnetic Dichroism
\cite{Bouldi1} experiments
performed on the beamlines at the Advanced Photon Source
(APS)\cite{APS}.
In particular this work was started for use on the beamlines at Beamline 4-ID-C
\cite{FOUR-IDC}
Attempts have been made to generalize this code so that it may be extended for 
use at other sources.  As this is tested at more beamlines at the APS, more 
documentation will be added to aid in expanding for use elsewhere.

\section{Software requirements}\label{softwareReqs}
This software was developed using Python\textsuperscript{\textregistered}
\cite{PYTHONWEB}, \cite{PYTHONLANGREF}, along with vaqrious standard Python
Packages that will be described in the installation section.  As is common with
Python, a 'packaged' distribution of Python was used in the development of
\textbf{\textit{xcirculardichro}} in order to ease the installation of many of
the required packages used in writing this program.  In particular, the Anaconda
\cite{ANACONDAPYTHON}Distribution of python was used here.  While Anaconda, Inc.
does provide some versions of it's distribution for sale, most users, and
certainly the core needs of this software, will require the version that can be
freely downloaded.

 
\chapter{Intallation}\label{chap:installation}
As explained in the previous chapter, this program was developed using the
Anaconda Distribution of Python.  Although it is probable that many other
packaged distributions of Python will work with this program, the installation
instructions found here will focus on installing
\textbf{\textit{xcirculardichro}} using Anaconda Python.

\section{Installing Anaconda Python}\label{sec:installingAnaconda}
The first step described here is the installation of Anaconda Python.  If the
user already has Anaconda (or another packaged distribution of Python)
installed, the user may want to skip to the next section on installing the
required Python packages.
At the time of this writing the current version of Anaconda Python is 5.0.1. 
Please note that this version number does not correspond to the actual version
of Python but to the combined set of packages that are contained in this
distribution from Anaconda, Inc.  The user should proceed to the Anaconda
webpage \url{https://www.anaconda.com} and then click on the "Download" link on
that page.  Following this link will lead the user to links to download a
version for the platform (Windows, Mac, Linux) on which they are running.
\newline
\vspace{3mm}


  \Fcolorbox{Black}{SpringGreen}%
  {
   \begin{minipage}{5.0in}
      If you desire to download for a different platform then follow the link
      for other versions.
   \end{minipage}}

\vspace{3mm}

\noindent Note that there are typically at least two versions of Anaconda to
chose from.  Although Version 3 of Python has been available for some time
Version 2 is still fairly common.  At this time, the Anaconda distribution
provides downloads configured for both Python 2.7 and 3.6.  If you do not have
any other Python versions installed, then I would recommend downloading \&
installing the 3.6 distribution.  \textbf{\textit{xcirculardichro}} was written
with Python 3.6.
If you are working with a fairly recent version of Anaconda Python which was
installed from a Python 2.7 installer you will likely not need to install
another version.  At this point look over material in Appendix A concerning
Anaconda and Python Environments.  In fact, if you are using another Python
distribution, some of this material may be helpful.  In a nutshell, Python
Environments allow you to set up a 'sandbox' where you can set up Python with a
set of modules and possibly (Anaconda allows this) a different version of
Python.  It will then allow you to activate/deactivate this version as needed.

\section{Installing Required Packages}\label{sec:installingReqPackages}
A number of common support packages are needed by
\textbf{\textit{xcirculardichro}}.  These include, but are not limited to:

\begin{itemize}
\itemsep0em
\item From the General Python Community

\begin{itemize}
\itemsep0em
\item PyQt (5.6)
\item matplotlib (2.1.0)
\item h5py (2.7.1)
\end{itemize}

\item From the Scientific/X-Ray Community

\begin{itemize}
\itemsep0em
\item spec2nexus
\item specguiutils
\end{itemize}

\end{itemize}

\noindent
There are a number of different ways to install python packages.  Past
installing the Anaconda distribution, you will mostly need to know how to
install extra packages from Anaconda using the 'conda' program and installing
from the Python Package Index\cite{PYTHONPACKAGEINDEX} using 'pip'
The packages from the General Python community can be installed using the conda
command.  To do this install run

conda install $<$package$>$
\newline
\noindent
For packages from the scientific comunity can be installed with the pip command
as

pip install $<$package$>$


\noindent
Also to install xcirculardichro

pip install xcirculardichro

\chapter{Running \textbf{\textit{xcirculardichro}}}\label{chap:running}

\section{Launching \textbf{\textit{xcirculardichro}}}\label{sec:launching}
To launch xcirculardichro run the command:

xcirculardichro    (on Linux/Mac)

or 

xcirculardichro    (on Windows)

You should see something like this: \\
\begin{figure}
\includegraphics[width=5in]{images/xmcd_afterFileOpen.png}
\caption{Image just after opening, the program area is divided into three
panes.  The pane on the left will show open files in a tree view, the middle
pane will show scan information for the selected file and the right is a graph
window that will show the selected scan data.}
\end{figure}

\section{Opening data file}
At this point this program is designed to work with \textbf{spec} data files
that are written at sector 4-ID at the Advanced Photon Source at Argonne
National Laboratory.  To open a file select \textbf{Open} from the \textbf{File}
menu and you should see an item appear selected on the tree view on the left and
information about the scans for that file in the middle.  As you select
data from the scans, plot of the selected data should show up in the pane on the
right.  It is possible to change the relative widths of these three
sections, to enhance selection or views, by dragging selectors between the
sections.  Note that some graphical elements will appear different on different
computer platforms (Windows, Mac \& Linux).  \\
\\
\begin{figure}
\includegraphics[width=5in]{images/xmcd_afterFileOpen.png}
\caption {After opening a file}
\end{figure}
\\
It is possible to open multiple \textbf{spec} files for analysis at one time. 
It is currently only possible to select one \textbf{spec} files for
manipulation.  To select a particular spec file, check the box next to the file
name.  Later we will see that it is possible to select subsets of data and place
them in an intermediate dataset which will also show up in the tree view.  It is
possible to select multiple intermediate data sets at one time, and display
these together on the graphs.

In the top of the middle pane is a table which will allow selection of scans
from the /textit{spec} file.  A pulldown menu above the list shows the different
scan types in the file or displaying all of the scans types at one time.  While
all scan types are shown, it is only possible to select one scan at a time. 
When a particular scan type is shown multiple selection is enabled but selection
of scans with a different number of scan points will raise a flag since the
selection will attempt to average the selected sets in many cases.  Options for
particular scan types will provide different options in the lower part of the
middle pane.  In particular, a different set of pull down options near the
middle may show and a different set of 'counters' (named data collected during
the scan) will be selected by default.  As we move into showing some of the
features for working with the data, we will demonstrate this using the
non-lockin data from 4-ID-C which has a scan type of \textit{qxdichro}.

One last feature of the middle section refers back to the table of available
scans.  When a \textbf{spec} file is selected, it is possible to add more data
to the table.  Selecting the \textbf{SelectBrowserParameters} from the
\textbf{View} menu will bring up a list of the positioner parameters (defined
by the \#On fields in the \textbf{spec} file)  as shown below.  Available
parameters will show on the left and a box for selected parameters is on the
right.  Parameters can be added or removed from either list by selecting the
desired parameter and pushing the appropriate button in the middle.

\begin{figure}
\includegraphics[width=5in]{images/xmcd_selectBrowserParameters.png}
\caption {Control for selecting positioners to display in the scan browser}
\end{figure}

When the selection of positioners is completed, the scan browser will have a new
column for each selected parameter and the value associated with each parameter
will be shown in the table.

\begin{figure}
\includegraphics[width=5in]{images/xmcd_scanBrowserWithPositioners.png}
\caption {Scan browser with extra positioners}
\label{fig:selectedScanBrowser}
\end{figure}

\section{Selecting Data}\label{sec:selectingData}
Now that we have some of the basics out of the way we can discuss more about
selecting and viewing data.  Again, for the purpose of consistancy, we will show
the non-lockin data as defined by qxdichro scan types.  If we load a run with
this type of data and then select qxdichro in the \textbf{Scan Type} pull down
then we will see all of the scans of this type in the scan browser.  This is as
shown in Figure~\ref{fig:selectedScanBrowser}.  

Selecting scans of this type will cause a set of options to populate in the
middle section of this pane as 

\begin{figure}[htp]

\centering
\subfloat[]{
\includegraphics[width=5in ]{images/xmcdSelectionPlotData.png}
\label{Expand Plot Data}
}

\subfloat[]{
\includegraphics[width=5in ]{images/xmcdSelectionDataType.png}
\label{Expand Data Type}
}

\subfloat[]{
\includegraphics[width=5in ]{images/xmcdSelectionPlotType.png}
\label{Expand Plot Type}
}
\caption {Choices for selected scans}
\label{fig:selectionChoice}
\end{figure}

More information about this type of data will be given in Appendix
\ref{sec:non-lockinDataType} but we will discuss some of the contents here. For
non-lockin type of data the goal is to produce data for two types signals, the
X-ray Absorption Spectroscopy (XAS) and the X-ray Magnetic Circular Dichroism
XMCD.
Data here is recorded with separate signals for diferent helicities, marked
\textbf{+} or \textbf{-} for convenience.  For the particular data set given
here this data was collected as a function of Energy.  In an ideal world, the
signals collected the same way all of the time, the correct signals would be
selected automatically and ready for processing.  In practice however, the
actual signals collected that would produce XAS \& XMCD change for some reason
so we have provided a way to select the correct signal.  This is done in the
lower section of the middle pane.  For this type of data we have five signals to
select: Energy, D+, D-, M+ and M-.  D+/D- are detector signals and M+/M- are
monitor signals.  XAS+ \& XAS- are given by D/M (with correct +/-)  This
represents the XAS signal associated with the particular helicity.

We can see below that the lower part of the middle pane will allow us to select
the counter gathered in the scan can be assigned to each of these signals and
will start to show the XAS/XMCD data.

With a particular scan type selected, it is also possible to select multiple
scans.  In this case it is possible to select (See Figure
\ref{fig:selectionChoice}(a)) to show all datasets together with the average of
the set, just the selected data, or just the average of the selected data.
\begin{figure}
\includegraphics[width=5in]{images/xmcdSelectedDataWithPlot.png}
\caption {Scan browser with extra positioners}
\label{fig:selectedScanBrowser}
\end{figure}

\begin{figure}[htp]

\centering \subfloat[Plotting selected data and the average of the data]{
\includegraphics[width=4.0in]{images/xmcdMultipleSelectionPlotDataAndAverage.png}
\label{Select Data and Average}
}

\subfloat[Plotting only the selected data]{
\includegraphics[width=4.0in]{images/xmcdMultipleSelectionPlotDataOnly.png}
\label{Select Data Only}
}

\subfloat[Plotting only the average of the selected data]{
\includegraphics[width=4.0in]{images/xmcdMultipleSelectionPlotAverageOnly.png}
\label{Select Average Only}
}
\caption {Choices for selected scans}
\label{fig:selectionChoice}
\end{figure}

\section{Processing Data and Intermediate Datasets}
\subsection{Discussion of the end goal}
At this point, we can finally settle in to processing the data.  At this point
we have selected scans for processing, selected appropriate counters to be used
in generating XAS+/- and XMCD+/- (See Appendix \ref{dataDefinitions} for the
specifics on different data types).  We then combine the data from the different
helicities to get XAS and XMCD.

In a real experiment we will have two data sets of this type.  This type of data
is taken with a magnetic field applied to the sample and in practice the data
will be taken with the magnetic field in one direction and then another set
taken with the field applied in the opposite direction.  In order to combine the
data, we will need to apply some normalization on the data, select processed
data to be extracted into intermediate datasets and then combine the data from
different field directions.  These steps will be discussed in the next few
sections.

\subsection{Applying Step Normalization}\label{sec:stepNormalization}

\begin{figure}
\includegraphics[width=5in]{images/xmcdPlotMarkedWithOffsets.png}
\caption {Plot of XAS showing baseline offsets before and after absorption
edge}
\label{fig:xmcdStepNormalizeBefore}
\end{figure}

A main correction that needs to be applied to the values calcultated thus far
come from adjusting XAS and XMCD based on baseline offsets in XAS measurement.
In Figure \ref{fig:xmcdStepNormalizeBefore} we see the XAS data marked with a
couple of offsets from the baseline (0.0).  We can apply a corrections to both
XAS and XCMD based on this.  Given the pre-edge offset and post-edge offset we
can calculate a \textbf{step} between these two 

\begin{equation}
step = offset_{post} - offset_{pre}
\label{eq:stepNormalizeStep}
\end{equation}

Based on this \textbf{step}, we can calculate a step-normalized XAS as

\begin{equation}
XAS_{step-normalized} = \frac{(XAS_{averaged} - offset_pre)}{step}
\label{eq:stepNormalizeXAS}
\end{equation}

and a step-normalized XMCD as
 
\begin{equation}
XMCD_{step-normalized} = \frac{XMCD_{averaged}}{step}
\label{eq:stepNormalizeXMCD}
\end{equation}

Note here we assume that XAS and XMCD have up to this point are the average of a
number of measurements.  Hence the $XAS_{averaged}$ and $XMCD_{Averaged}$ This
is done to ensure that enough statistics for the measurement are made without
risk of losing data due to a problem in the sytem.

\begin{figure}[htp]

\centering
\subfloat[Plotting selected data and the average of the data]{
\includegraphics[width=4in]{images/xmcdPointSelectionControl.png}
\label{fig:pointSelectionControlWNoPointsSelected}
}

\subfloat[Plotting only the selected data]{
\includegraphics[width=4.0in]
{images/xmcdPointSelectionControlWPointsSelected.png}
\label{fig:pointSelectionControlWNoPointsSelected}
}

\caption {Choices for selected scans}
\label{fig:pointSelectionControl}
\end{figure}

To set up for making this correction we need to select points before and after
the absorption edge.   To do this, there is a control in the center pane, below
the counter selection table (See Figure
\ref{fig:pointSelectionControlWNoPointsSelected}).
Here there are two pulldown boxes to aid in setting up the point selection.  The
\textbf{PointSet to Select} pulldown allows selection of either the pre-edge or
post-edge points.  This means that if the pre-edge is selected, and the user
clicks on a point in the graph the data point is added to the \textbf{Pre-Edge
Indices} and it's value contributes to the Average.   For this the selected
points will be red.  To deselect a point simply click on it once again.   With
\textbf{Post-Edge} selected, points that are clicked on will be added to the
\textbf{Post-Edge Indicies} and the value will be added to the appropriate
average.  Note that points for the post-Edge should show in blue and a blue line
representing the average will be drawn over the range of selected points.  See
figures \ref{fig:pointSelectionControlWNoPointsSelected} and
\ref{fig:plotSelectedPoints}.
The \textbf{PointSelect Axis} pulldown selects data from the left axis or the right
axis is active for selection.  Normally, XAS is plottted on the Left axis so
this control is normally left set to \textbf{Left}.

\begin{figure}
\includegraphics[width=5in]
{images/xmcdPlotWithPointsSelectedForStepNormalization.png}
\caption {Plot shown with data points selected for step normalization}
\label{fig:plotSelectedPoints}
\end{figure}

Once points are selected for this correction, a new set of curves should appear
(sometimes this does not happen until another scan is selected) showing the
Step-Normalized Data.  Note that these plots will typically appear right over
the original plots.  An additional set of legends show at the bottom for the
corrected data (Figure \ref{fig:plotStepNormalizedData}).  The scale for the
corrected data shows on the outer axes.  Note that XAS will now appear closer to
the origin since we shifted by the pre-edge offset before scaling, and that the
axes scales are quite diferent due to the scaling factor from the step size.

\begin{figure}
\includegraphics[width=5in]
{images/xmcdPlotWithStepNormalizedData.png}
\caption {Plot shown with data points selected for step normalization}
\label{fig:plotStepNormalizedData}
\end{figure}

\subsection{Extracting Intermediate Data and Combining with Other Results}
\begin{figure}
\includegraphics[width=5in]
{images/xmcdMultipleSelectWithStepNormalization.png}
\caption {Multiple scans averages and step-normalized}
\label{fig:multipleSelectWithStepNormalization}
\end{figure}
Now that we have covered step-normalized data we can start to consider combining
data from the two different field directions.  At this point we assume that we
have collected a number of scans with each field direction.  We have selected
first all of the scans from one direction, averaging the result and then
performing a step normalization this set.  This is shown in Figure
\ref {fig:multipleSelectWithStepNormalization} We can then capture this result,
placing it into an intermediate dataset.  This is done by selecting
\textbf{Capture Current Corrected]} from the \textbf{Data} menu. Once this is
done a new element will appear in the tree view on the left pane.  We can
select this element and the 'scan' that appears in the scan browser in the
middle pane and we will see the plot of the data in the right pane as in Figure
\ref{fig:stepNormalizedInIntermediateSet}.  Now that we have the data from one
field direction processed we can follow the same steps for the second field
direction, ultimately placing a second intermediate dataset into the tree view.
Shown in Figure \ref{fig:stepNormalizedBothFieldsInIntermediateSet}.

\begin{figure}
\includegraphics[width=5in]
{images/xmcdStepNormalizedInIntermediateSet.png}
\caption {Step-normalized data in an intermediate dataset}
\label{fig:stepNormalizedInIntermediateSet}
\end{figure}

\begin{figure}
\includegraphics[width=5in]
{images/xmcdStepNormalizedBothFieldsInOwnIntermediateSet.png}
\caption {Step-normalized data from both field directions.  Each direction is in
an intermediate dataset}
\label{fig:stepNormalizedBothFieldsInIntermediateSet}
\end{figure}

The data as it appears puts the two step-normalized datasets together and simply
averages the two sets.  The final step for this kind of analysis is to combine
these results together, similar to the two helicities, averaging XAS results to
get the final XAS and subtracting the XMCD results to get XMCD.  This can be
done by selecting a new option in \textbf{PlotData} pulldown in the middle pane.
Since we are now dealing with two intermediate datasets we are presented with a
new set of options as shown in Figure
\ref{fig:plotDataSelectionOptionsIntermediateSet}.

\begin{figure}
\includegraphics[width=4in]
{images/xmcdSelectionParametersForIntermediateData.png}
\caption {Plot Data options for Intermediate Datasets}
\label{fig:plotDataSelectionOptionsIntermediateSet}
\end{figure}

\begin{figure}
\includegraphics[width=5in]
{images/xmcdPlotOfNormalizedDataFromBothFieldDirections.png}
\caption {Plot of data from both field directions with final normalized result}
\label{fig:plotnormalizedDataBothFields}
\end{figure}

\begin{appendices}
\chapter{Anaconda \& Python Environments}
\section{What is a Python Environment}
Python environments\cite{PYTHONVIRTUALENVIRONMENT} provide a means to create a separate
python space for running developing and running applications which require a different set 
of packages and/or different versions of packages.  This is a convenient 
method to deal with problems that often occur with when packages depend on 
differnt versions of another package.  It is also a convinient way to deal 
with the current problem that there are two major versions of Python itself
(2.7 and 3.x) and different applications may require one or the other of 
these versions.  In order to deal with these issues, Python has created a
system for defining a named Python environment in which to run a particular
application and a fairly simple mechanism to switch between these 
environments to run your applications.  Note that the Anaconda distribution of
Python includes it's own version of Python environments\cite{ANACONDAENVIRONMENT} and
the discussion of Python Environments covered here will use this version.

\section{Anaconda Environments}
The Anaconda distribution of Python currently provides two different 
downloads of their Python tools.  One of these is set up for Python 2.7 
and the other is set up for Python 3.6.  Our application
\textbf{xcirculardichro} requires Python 3.6 to run.  If you have already installed the Python 2.7 
version of Anaconda you \textbf{do not} need to install a separate version 
of Anaconda in order to run \textbf{xcirculardichro}.  This is because the
only difference between the 2.7 and 3.6 version of the distribution is that
the default environment provided by the install is either Python 2.7 or 
Python 3.6 respectively.  It is possible to create a separate environment
from either one which is configured for the other Python version (or possibly 
other versions in between).  If you have Python 2.7 installed, or if 
you want to create a separate environment to install and run
\textbf{xcirculardichro} you can start by running the command
\lstset{language=bash}
\begin{lstlisting}
conda create -name xcirculardichro -python=3.6
\end{lstlisting}
This you will then be prompted with some information about what packages
will be included and asked to accept.  On accepting this the system will 
create a directory in which to place the new environment which will include 
binaries (or links to them), libraries and packages for this environment. This
wiwill typically be placed in the user's home directory, under a subdirectory
\texttt{anaconda/envs/xcirculardichro} or
\texttt{anaconda3/envs/xcirculardichro}  depending on the installed version of
the anaconda tools).  To make this python environment the active python 
environment type the command

\lstset{language=bash}
\begin{lstlisting}
source activate xcirculardichro (for linux or Mac)
\end{lstlisting}

or

\lstset{language=bash}
\begin{lstlisting}
activate xcirculardichro (for Windows)
\end{lstlisting}

After setting up an environment in this manner you can then follow the
instructions for installing an application, such as \textbf{xcirculardichro}. 
Note that at this point you have a version of python with a minimal number of
packages (unless you have configured conda to create environments with a
default set of packages) so after activating the environment you will need to
install packages needed by \textbf{xcirculardichro} and their depenancies using 
either 

\lstset{language = bash}
\begin{lstlisting}
conda install packagename
\end{lstlisting}

or 

\begin{lstlisting}
pip install packagename
\end{lstlisting}

as appropriate. 

Just to reiterate here, there is nothing special here about the Anaconda Python
distribution except the authors familiarity with this and it's use at the 
Advanced Photon Source.  It should be possible to install
\textbf{xcirculardichro} using any python distribution and that concept of 
python environments is not unique to anaconda.  Anaconda does seem to have 
some modifications to the standard python environment but much of the above
discussion whould hold for most modern Python distributions.
\chapter{Scan Type Definitions\label{dataDefinitions}}
\section{default scan type\label{sec:defaultDataType}}
The default data type is simply a type to allow display of data with unknown
characteristices.  This type allows selection of an x axis and two signals. 
This data will simply be treated as generic x, y data pairs.  No specific
calculation will be performed on this data type.  If multiple scans are
selected, then the data will be averaged over the scans.  Data will be displayed
on the graph view.

\section{non-lockin type (qxdichro, kepdichro, tempdichro,
etc)}\label{sec:non-lockinDataType}
\begin{figure}
\includegraphics[width=5in]
{images/xmcdCounterSelectForNonLockin.png}
\caption {Selection of counters for non-lockin types}
\label{fig:counterSelectNonLockin}
\end{figure}

This data type is taken such that data from each helicity is taken in a separate
data set and must be combined to give the overall XAS and XMCD for a particular
field direction.  This collection type assumes that the data is collected as a
detector/monitor pair.  When this scan type if data is selected in the scan
browser then the counter selector in the lower part of the middle pane will
allow the user to select the data source (from collected scan fields) for the
dependant variable (E, Temp, etc) and D+, D-, M+ and M- (detector and monitor
for each helicity) as in Figure \ref{fig:counterSelectNonLockin}.  At Sector
4-ID, scan files usually place the counters associated with the detector and
monitor near the end of the counter list.  Usually the last columns (in the scan
file) are sum, dist and flip.  The four columns preceeding these are
M\textsubscript{+}, M\textsubscript{-}, D\textsubscript{+} and
D\textsubscript{-}.  In order to facilitate long collection times without the
risk of data loss, a number of measurements are typically repeated and stored in
multiple scans in the SPEC file.  One feature of \textbf{xcirculardichro} is
that it can automatically average multiple scans simply by selecting the desired
scans.  In order to do multiple selection you will need to select a particular
data type (such as qxdichro) and then select multiple scans with the same number
of data points.

From these measurements we will be able to calculate XAS
for each of the helicities as Equation \ref{eq:xasEachHelicityNonLockin} or
{eq:xasEachHelicityTransmisssionNonLockin}:

\begin{equation}
XAS_+ = \frac{D_+}{M_+} \quad\mathrm{   and   }\quad XAS_- = \frac{D_-}{M_-}
\label{eq:xasEachHelicityNonLockin}
\end{equation}

for flourescence measurement, or

\begin{equation}
XAS_+ = log_e\frac{M_+}{D_+} \quad\mathrm{   and   }\quad XAS_-
= log_e\frac{M_-}{D_-}
\label{eq:xasEachHelicityTransmisssionNonLockin}
\end{equation}

for transmission measurement.

From the measurements of XAS for each helicity we can then calculate XAS and
XMCD for a particular field direction $\vec{B}$ as 

\begin{equation}
\begin{split}
XAS_{\vec{B}} = \frac{XAS_{+(\vec{B})} + XAS_{-(\vec{B})}}{2} 
\\
 XMCD_{\vec{B}} = XAS_{+(\vec{B})} - XAS_{-(\vec{B})} .
\label{eq:xasFieldDirectionFlourecenceNonLockin}
\end{split}
\end{equation}

These measurements are then normalized through a process called step
normalization.  In order to step normalize XAS and XMCD calculated for a
particular field direction, one would measusre a zero offset below the
absorption edge and step offset after the absorption edge as in Figure
\ref{fig:xmcdStepNormalizeBefore}.  Normalization would be calculated from these
offsets as given by Equations \ref{eq:stepNormalizeStep},
\ref{eq:stepNormalizeXAS} and \ref{eq:stepNormalizeXMCD}

A measurement is made in opposite field direction $\vec{B}$ and $-\vec{B}$ and a
final normalized XAS and XMCD are calculated as

\begin{equation}
\begin{split}
XAS_final = \frac{XAS_{\vec{B}} + XAS_{-\vec{B}}}{2}
\\
XMCD_final = XAS_{\vec{B}} - XAS_{-\vec{B}}
\end{split}
\end{equation}

In practice, processing these values is outlined in \ref{sec:stepNormalization}



\section{lock-in type (qxscan)}\label{sec:lockinDataType}
For this type of data, the detectors already contain XAS and XMCD data that is
polarized averaged.  This means that many of the steps outlined for the
Non-Lockin data described in Section \ref{sec:non-lockinDataType}.  Lock-in data
is normally collected at Sector 4-ID in \textbf{SPEC} files and the XAS and XMCD
data are typically collected as the last two columns of the file and will be labeled as
lockDC and lockACfix.  In order to perform longer acquisition without the risk
of data loss, data is normally collected in a series of small steps collected in
multiple scans in the \textbf{SPEC} file.  To deal with this,
\textbf{xcirculardichro} is able to automaticlly average data from multiple
scans simply by selecting multiple scans in the scan browser (see Sectiom
\ref{dataSelection}).  It is likely that it will be necessary to step normalize
this data as shown in Section \ref{sec:stepNormalization}.  From here it is a
matter of combining data from different scans by first extracting step
normalized data for XAS and XMCD for each field direction as described in
Section 




\end{appendices}

\printbibliography

\end{document}
 